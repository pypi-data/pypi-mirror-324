\documentclass{article}
\usepackage[utf8]{inputenc}
\usepackage{hyperref}
\usepackage{listings}
\usepackage{xcolor} % For custom colors
\usepackage{sectsty} % For customizing section fonts
% \subsectionfont{\raggedright} % Allow line breaks and ragged right alignment             
\usepackage{geometry}
% \geometry{margin=1in}
\usepackage{graphicx}
                
\title{TexTOM - Manual}
\author{Moritz Frewein, Moritz Stammer, Marc Allain, Tilman Grünewald}
\begin{document}
\maketitle
\label{toc}
\tableofcontents

% Define a custom style for docstrings
\lstdefinelanguage{docstring}{
    basicstyle=\ttfamily\small, % Monospaced font
    backgroundcolor=\color[HTML]{F5F5F5}, % Light gray background
    frame=single, % Border around the docstring
    rulecolor=\color[HTML]{D6D6D6}, % Border color
    keywordstyle=\color{blue}, % Optional: color for keywords
    breaklines=true, % Wrap long lines
}
                
\subsectionfont{\large\ttfamily\raggedright}

\chapter{Introduction}

\code{djehuty} is the data repository system developed by
4TU.ResearchData and Nikhef.  The name finds its inspiration in
\href{https://en.wikipedia.org/wiki/Thoth}{Thoth}, the Egyptian
entity that introduced the idea of writing.

\section{Obtaining the source code}
\label{sec:obtaining-tarball}

  \begin{sloppypar}
  The source code can be downloaded at the
  \href{https://github.com/4TUResearchData/djehuty/releases}%
  {Releases}%
  \footnote{\url{https://github.com/4TUResearchData/djehuty/releases}}
  page.  Make sure to download the {\fontfamily{\ttdefault}\selectfont
    djehuty-\djehutyversion{}.tar.gz} file.
  \end{sloppypar}

  Or, directly download the tarball using the command-line:

\begin{lstlisting}[language=bash]
curl -LO https://github.com/4TUResearchData/djehuty/releases/\(@*\\*@)download/v(@*\djehutyversion{}*@)/djehuty-(@*\djehutyversion{}*@).tar.gz
\end{lstlisting}

  After obtaining the tarball, it can be unpacked using the \t{tar}
  command:

\begin{lstlisting}[language=bash]
tar zxvf djehuty-(@*\djehutyversion{}*@).tar.gz
\end{lstlisting}

\section{Installing the prerequisites}
\label{sec:prerequisites}

  The \code{djehuty} program needs Python (version 3.9 or higher) and
  Git to be installed.  Additionally, a couple of Python packages need
  to be installed.  The following sections describe installing the
  prerequisites on various GNU/Linux distributions.  To put the software in
  the context of its environment, figure \ref{fig:references-graph} displays
  the complete run-time dependencies from \t{djehuty} to \t{glibc}.

  \includefigure{references-graph}{Run-time references when constructed with
    the packages from GNU Guix.}

  The web service of \code{djehuty} stores its information in a SPARQL 1.1
  \citep{sparql-11} endpoint.  We recommend either
  \href{https://blazegraph.com/}{Blazegraph}%
  \footnote{https://blazegraph.com/}
  or \href{http://vos.openlinksw.com/owiki/wiki/VOS}%
  {Virtuoso open-source edition}%
  \footnote{http://vos.openlinksw.com/owiki/wiki/VOS}.

\section{Installation instructions}

  After obtaining the source code (see section \refer{sec:obtaining-tarball})
  and installing the required tools (see section \refer{sec:prerequisites}),
  building involves running the following commands:

\begin{lstlisting}[language=bash]
cd djehuty-(@*\djehutyversion{}*@)
autoreconf -vif # Only needed if the "./configure" step does not work.
./configure
make
make install
\end{lstlisting}

  To run the \t{make install} command, super user privileges may be
  required.  Specify a \t{-{}-prefix} to the \t{configure}
  script to install the tools to a user-writeable location to avoid
  needing super user privileges.

  After installation, the \t{djehuty} program will be available.

\section{Pre-built containers}

  4TU.ResearchData provides Docker container images as a convenience service
  for each monthly \t{djehuty} release.  The following table outlines the
  meaning of each image provided.  The images are published to
  \href{https://hub.docker.com/r/4turesearchdata/djehuty}{Docker Hub}%
  \footnote{\dhref{https://hub.docker.com/r/4turesearchdata/djehuty}}.

\begin{tabularx}{\textwidth}{*{1}{!{\VRule[-1pt]}l}!{\VRule[-1pt]}X}
  \headrow
  \textbf{Image tag}  & \textbf{Description}\\
  \t{devel}           & Image meant for development purposes.  Before it
                        executes the \t{djehuty} command it checks out the
                        latest codebase.  So re-running the same container
                        image may result in running a different version of
                        \t{djehuty}.\\
  \t{latest}          & This image points to the latest \t{djehuty} release.
                        It does not automatically update the \t{djehuty}
                        codebase.\\
  \t{XX.X}            & 4TU.ResearchData releases a version each month where
                        the number before the dot refers to the year and the
                        number after the dot refers to the month.  Use a
                        specific version image when you want to upgrade at
                        your own pace.
\end{tabularx}

  To build the container images for yourself, see the build instructions in
  the \file{docker/Dockerfile} file.

\section{Functions}\label{sec:functions}
\subsection*{\texttt{set\_path(path)}}
\label{fun:setpath}
\addcontentsline{toc}{subsection}{set\_path}

\begin{lstlisting}[language=docstring]
Set the path where integrated data and analysis is stored

Parameters
----------
path : str
    full path to the directory, must contain a folder '/data_integrated'
\end{lstlisting}

\begin{flushright}

\hyperref[toc]{ToC}

\end{flushright}

\input{functions/set_path}

\vspace{5mm}

\hrule

\subsection*{\texttt{check\_state()}}
\label{fun:checkstate}
\addcontentsline{toc}{subsection}{check\_state}

\begin{lstlisting}[language=docstring]
Prints in terminal which parts of the reconstruction are ready
    
\end{lstlisting}

\begin{flushright}

\hyperref[toc]{ToC}

\end{flushright}

\input{functions/check_state}

\vspace{5mm}

\hrule

\subsection*{\texttt{integrate()}}
\label{fun:integrate}
\addcontentsline{toc}{subsection}{integrate}

\begin{lstlisting}[language=docstring]
Integrates raw 2D diffraction data via pyFAI

All necessary input will be handled via the file integration_parameters.py
\end{lstlisting}

\begin{flushright}

\hyperref[toc]{ToC}

\end{flushright}

\input{functions/integrate}

\vspace{5mm}

\hrule

\subsection*{\texttt{align\_data(pattern='.h5', sub\_data='data\_integrated', q\_index\_range=(0, 5), q\_range=False, crop\_image=False, mode='optical\_flow', redo\_import=False, flip\_fov=False, regroup\_max=16, align\_horizontal=True, align\_vertical=True, pre\_rec\_it=5, pre\_max\_it=5, last\_rec\_it=40, last\_max\_it=5)}}
\label{fun:aligndata}
\addcontentsline{toc}{subsection}{align\_data}

\begin{lstlisting}[language=docstring]
Align data using the Mumott optical flow alignment

Requires that data has been integrated and that sample_dir contains
a subfolder with data

Parameters
----------
pattern : str, optional
    substring contained in all files you want to use, by default '.h5'
sub_data : str, optional
    subfolder containing the data, by default 'data_integrated'
q_index_range : tuple, optional
    determines which q-values are used for alignment (sums over them), by default (0,5)
q_range : tuple, optional
    give the q-range in nm instead of indices e.g. (15.8,18.1), by default False
crop_image : bool or tuple of int, optional
    give the range you want to use in x and y, e.g. ((0,-1),(10,-10)), by default False
mode : str, optional
    choose alignment mode, 'optical_flow' or 'phase_matching', by default 'optical_flow'
redo_import : bool, optional
    set True if you want to recalculate data_mumott.h5, by default False
flip_fov : bool, optional
    only to be used if the fov is in the wrong order in the integrated
    data files, by default False
regroup_max : int, optional
    maximum size of groups when downsampling for faster processing, by default 16
align_horizontal : bool, optional
    align your data horizontally, by default True
align_vertical : bool, optional
    align your data vertically, by default True
pre_rec_it : int, optional
    reconstruciton iterations for downsampled data, by default 5
pre_max_it : int, optional
    alignment iterations for downsampled data, by default 5
last_rec_it : int, optional
    reconstruciton iterations for full data, by default 40
last_max_it : int, optional
    alignment iterations for full data, by default 5
\end{lstlisting}

\begin{flushright}

\hyperref[toc]{ToC}

\end{flushright}

\input{functions/align_data}

\vspace{5mm}

\hrule

\subsection*{\texttt{check\_alignment\_consistency()}}
\label{fun:checkalignmentconsistency}
\addcontentsline{toc}{subsection}{check\_alignment\_consistency}

\begin{lstlisting}[language=docstring]
Plots the squared residuals between data and the projected tomograms

    
\end{lstlisting}

\begin{flushright}

\hyperref[toc]{ToC}

\end{flushright}

\input{functions/check_alignment_consistency}

\vspace{5mm}

\hrule

\subsection*{\texttt{check\_alignment\_projection(g=0)}}
\label{fun:checkalignmentprojection}
\addcontentsline{toc}{subsection}{check\_alignment\_projection}

\begin{lstlisting}[language=docstring]
Plots the data and the projected tomogram of projection g

Parameters
----------
g : int, optional
    projection running index, by default 0
\end{lstlisting}

\begin{flushright}

\hyperref[toc]{ToC}

\end{flushright}

\input{functions/check_alignment_projection}

\vspace{5mm}

\hrule

\subsection*{\texttt{make\_model()}}
\label{fun:makemodel}
\addcontentsline{toc}{subsection}{make\_model}

\begin{lstlisting}[language=docstring]
Calculates the TexTOM model for reconstructions

Is automatically performed by the functions that require it
\end{lstlisting}

\begin{flushright}

\hyperref[toc]{ToC}

\end{flushright}

\input{functions/make_model}

\vspace{5mm}

\hrule

\subsection*{\texttt{preprocess\_data(pattern='.h5', flip\_fov=False, baselines=1, use\_ion=True)}}
\label{fun:preprocessdata}
\addcontentsline{toc}{subsection}{preprocess\_data}

\begin{lstlisting}[language=docstring]
Loads integrated data and pre-processes them for TexTOM

Parameters
----------
pattern : str, optional
    substring contained in all files you want to use, by default '.h5'
flip_fov : bool, optional
    only to be used if the fov is in the wrong order in the integrated
    data files, by default False
baselines : bool, optional
    choose if you want to draw polynomial baselines
    set the polynomial order in the argument or False, by default 1
use_ion : bool, optional
    choose if you want to normalize data by the field 'ion' in the 
    data files, by default True
\end{lstlisting}

\begin{flushright}

\hyperref[toc]{ToC}

\end{flushright}

\input{functions/preprocess_data}

\vspace{5mm}

\hrule

\subsection*{\texttt{make\_fit(redo=True)}}
\label{fun:makefit}
\addcontentsline{toc}{subsection}{make\_fit}

\begin{lstlisting}[language=docstring]
Initializes a TexTOM fit object for reconstructions

Is automatically performed by the functions that require it

Parameters
----------
redo : bool, optional
    set True for recalculating, by default True
\end{lstlisting}

\begin{flushright}

\hyperref[toc]{ToC}

\end{flushright}

\input{functions/make_fit}

\vspace{5mm}

\hrule

\subsection*{\texttt{optimize(order=0, mode=0, proj='full', zero\_peak=None, redo\_fit=False, tol=0.001, minstep=1e-09, itermax=3000, alg='quadratic', save\_h5=True)}}
\label{fun:optimize}
\addcontentsline{toc}{subsection}{optimize}

\begin{lstlisting}[language=docstring]
Performs a single TexTOM parameter optimization

Parameters
----------
order : int, optional
    maximum sHSH order to be used, by default 0
mode : int, optional
    set 0 for only optimizing order 0, 1 for highest order, 2 for all,
    by default 0
proj : str, optional
    choose projections to be optimized: 'full', 'half', 'third', 'notilt', 
    by default 'full'
zero_peak : int or None
    index of the peak you want to use for 0-order fitting (should be as
    isotropic as possible), if None uses the whole dataset, default None
redo_fit : bool, optional
    recalculate the fit object, by default False
tol : float, optional
    tolerance for precision break criterion, by default 1e-3
minstep : float, optional
    minimum stepsize in line search, by default 1e-9
itermax : int, optional
    maximum number of iterations, by default 3000
alg : str, optional
    choose algorithm between 'backtracking', 'simple', 'quadratic', 
    by default 'quadratic'
save_h5 : bool, optional
    choose if you want to save the result to the directory analysis/fits, 
    by default True    
\end{lstlisting}

\begin{flushright}

\hyperref[toc]{ToC}

\end{flushright}

\input{functions/optimize}

\vspace{5mm}

\hrule

\subsection*{\texttt{optimize\_auto(max\_order=8, start\_order=None, zero\_peak=None, tol\_0=1e-07, tol\_1=0.001, tol\_2=0.0001, minstep\_0=1e-09, minstep\_1=1e-09, minstep\_2=1e-09, projections='full', alg='quadratic', adj\_scal=False, redo\_fit=False)}}
\label{fun:optimizeauto}
\addcontentsline{toc}{subsection}{optimize\_auto}

\begin{lstlisting}[language=docstring]
Automated TexTOM reconstruction workflow

Parameters
----------
max_order : int, optional
    maximum HSH order to be used, by default 8    
start_order : int or None, optional
    lowest order to be fitted, if None continues where you are standing, 
    by default None
zero_peak : int or None
    index of the peak you want to use for 0-order fitting (should be as
    isotropic as possible), if None uses the whole dataset, default None
redo_fit : bool, optional
    recalculate the fit object, by default False
proj : str, optional
    choose projections to be optimized: 'full', 'half', 'third', 'notilt', by default 'full'
alg : str, optional
    choose algorithm between 'backtracking', 'simple', 'quadratic', 
    by default 'quadratic'
\end{lstlisting}

\begin{flushright}

\hyperref[toc]{ToC}

\end{flushright}

\input{functions/optimize_auto}

\vspace{5mm}

\hrule

\subsection*{\texttt{adjust\_data\_scaling()}}
\label{fun:adjustdatascaling}
\addcontentsline{toc}{subsection}{adjust\_data\_scaling}

\begin{lstlisting}[language=docstring]
Reestimates the data based on the assumption that normalization constants
contain noise. To be used after fitting the 0th order
\end{lstlisting}

\begin{flushright}

\hyperref[toc]{ToC}

\end{flushright}

\input{functions/adjust_data_scaling}

\vspace{5mm}

\hrule

\subsection*{\texttt{list\_opt()}}
\label{fun:listopt}
\addcontentsline{toc}{subsection}{list\_opt}

\begin{lstlisting}[language=docstring]
Shows all stored optimizations
    
\end{lstlisting}

\begin{flushright}

\hyperref[toc]{ToC}

\end{flushright}

\input{functions/list_opt}

\vspace{5mm}

\hrule

\subsection*{\texttt{load\_opt(h5path='last')}}
\label{fun:loadopt}
\addcontentsline{toc}{subsection}{load\_opt}

\begin{lstlisting}[language=docstring]
Loads a previous Textom optimization into memory
seful: load_opt(results['optimization'])

Parameters
----------
h5path : str, optional
    filepath, just filename or full path
    if 'last', uses the youngest file is used in analysis/fits/, 
    by default 'last'
\end{lstlisting}

\begin{flushright}

\hyperref[toc]{ToC}

\end{flushright}

\input{functions/load_opt}

\vspace{5mm}

\hrule

\subsection*{\texttt{check\_lossfunction()}}
\label{fun:checklossfunction}
\addcontentsline{toc}{subsection}{check\_lossfunction}

\begin{lstlisting}[language=docstring]
No docstring available.
\end{lstlisting}

\begin{flushright}

\hyperref[toc]{ToC}

\end{flushright}

\input{functions/check_lossfunction}

\vspace{5mm}

\hrule

\subsection*{\texttt{check\_fit\_average()}}
\label{fun:checkfitaverage}
\addcontentsline{toc}{subsection}{check\_fit\_average}

\begin{lstlisting}[language=docstring]
Plots the reconstructed average intensity for each projection with data

Parameters
----------
\end{lstlisting}

\begin{flushright}

\hyperref[toc]{ToC}

\end{flushright}

\input{functions/check_fit_average}

\vspace{5mm}

\hrule

\subsection*{\texttt{check\_fit\_random(N=10, mode='line')}}
\label{fun:checkfitrandom}
\addcontentsline{toc}{subsection}{check\_fit\_random}

\begin{lstlisting}[language=docstring]
Generates TexTOM reconstructions and plots them with data for random points

Parameters
----------
N : int, optional
    Number of images created, by default 10    
mode : str, optional
    plotting mode, 'line' or 'color', by default line
\end{lstlisting}

\begin{flushright}

\hyperref[toc]{ToC}

\end{flushright}

\input{functions/check_fit_random}

\vspace{5mm}

\hrule

\subsection*{\texttt{check\_residuals()}}
\label{fun:checkresiduals}
\addcontentsline{toc}{subsection}{check\_residuals}

\begin{lstlisting}[language=docstring]
Plots the squared residuals summed over each projection
    
\end{lstlisting}

\begin{flushright}

\hyperref[toc]{ToC}

\end{flushright}

\input{functions/check_residuals}

\vspace{5mm}

\hrule

\subsection*{\texttt{check\_projections\_average(G=None)}}
\label{fun:checkprojectionsaverage}
\addcontentsline{toc}{subsection}{check\_projections\_average}

\begin{lstlisting}[language=docstring]
Plots the reconstructed average intensity for chosen projections with data

Parameters
----------
G : int or ndarray or None, optional
    projection indices, if None takes 10 equidistant ones, by default None
\end{lstlisting}

\begin{flushright}

\hyperref[toc]{ToC}

\end{flushright}

\input{functions/check_projections_average}

\vspace{5mm}

\hrule

\subsection*{\texttt{check\_projections\_residuals(G=None)}}
\label{fun:checkprojectionsresiduals}
\addcontentsline{toc}{subsection}{check\_projections\_residuals}

\begin{lstlisting}[language=docstring]
Plots the residuals per pix3l for chosen projections with data

Parameters
----------
G : int or ndarray or None, optional
    projection indices, if None takes 10 equidistant ones, by default None
\end{lstlisting}

\begin{flushright}

\hyperref[toc]{ToC}

\end{flushright}

\input{functions/check_projections_residuals}

\vspace{5mm}

\hrule

\subsection*{\texttt{check\_projections\_orientations(G=None)}}
\label{fun:checkprojectionsorientations}
\addcontentsline{toc}{subsection}{check\_projections\_orientations}

\begin{lstlisting}[language=docstring]
Plots the reconstructed average orientations for chosen projections with data

Parameters
----------
G : int or ndarray or None, optional
    projection indices, if None takes 10 equidistant ones, by default None
\end{lstlisting}

\begin{flushright}

\hyperref[toc]{ToC}

\end{flushright}

\input{functions/check_projections_orientations}

\vspace{5mm}

\hrule

\subsection*{\texttt{calculate\_orientation\_statistics()}}
\label{fun:calculateorientationstatistics}
\addcontentsline{toc}{subsection}{calculate\_orientation\_statistics}

\begin{lstlisting}[language=docstring]
Calculates prefered orientations and stds and saves them to results dict

    
\end{lstlisting}

\begin{flushright}

\hyperref[toc]{ToC}

\end{flushright}

\input{functions/calculate_orientation_statistics}

\vspace{5mm}

\hrule

\subsection*{\texttt{calculate\_segments(thresh=10, min\_segment\_size=30, max\_segments\_number=31)}}
\label{fun:calculatesegments}
\addcontentsline{toc}{subsection}{calculate\_segments}

\begin{lstlisting}[language=docstring]
Segments the sample based on misorientation borders

Parameters
----------
thresh : float, optional
    misorientation angle threshold inside segment in degree, by default 10
min_segment_size : int, optional
    minimum number of voxels in segment, by default 30
max_segments_number : int, optional
    maximum number of segments (ordered by size), by default 32
\end{lstlisting}

\begin{flushright}

\hyperref[toc]{ToC}

\end{flushright}

\input{functions/calculate_segments}

\vspace{5mm}

\hrule

\subsection*{\texttt{show\_volume(data='scaling', plane='z', colormap='inferno', cut=1, save=False, show=True)}}
\label{fun:showvolume}
\addcontentsline{toc}{subsection}{show\_volume}

\begin{lstlisting}[language=docstring]
Visualizes the whole sample by slices, colored by a value of your choice

Parameters
----------
data : str or list, optional
    name of one entry in the results dict or list of entries, 
    by default 'scaling'
plane : str, optional
    sliceplane 'x'/'y'/'z', by default 'z'
colormap : str, optional
    identifier of matplotlib colormap, default 'inferno'
    https://matplotlib.org/stable/users/explain/colors/colormaps.html
cut : int, optional
    cut colorscale at upper and lower percentile, by default 0.1
save : bool, optional
    saves tomogram as .gif to results/images/, by default False
show : bool, optional
    open the figure upon calling the function, by default True
\end{lstlisting}

\begin{flushright}

\hyperref[toc]{ToC}

\end{flushright}

\input{functions/show_volume}

\vspace{5mm}

\hrule

\subsection*{\texttt{show\_slice\_ipf(h, plane='z')}}
\label{fun:showsliceipf}
\addcontentsline{toc}{subsection}{show\_slice\_ipf}

\begin{lstlisting}[language=docstring]
Plots an inverse pole figure of a sample slice

Parameters
----------
h : int
    height of the slice
plane : str, optional
    slice direction: x/y/z, by default 'z'
\end{lstlisting}

\begin{flushright}

\hyperref[toc]{ToC}

\end{flushright}

\input{functions/show_slice_ipf}

\vspace{5mm}

\hrule

\subsection*{\texttt{show\_volume\_ipf(plane='z', save=False, show=True)}}
\label{fun:showvolumeipf}
\addcontentsline{toc}{subsection}{show\_volume\_ipf}

\begin{lstlisting}[language=docstring]
Plots inverse pole figures as a tomogram with a slider to scroll through the sample

Parameters
----------
plane : str, optional
    slice direction: x/y/z, by default 'z'
save : bool, optional
    saves tomogram as .gif to results/images/, by default False
show : bool, optional
    open the figure upon calling the function, by default True
\end{lstlisting}

\begin{flushright}

\hyperref[toc]{ToC}

\end{flushright}

\input{functions/show_volume_ipf}

\vspace{5mm}

\hrule

\subsection*{\texttt{show\_histogram(x, nbins=50, cut=0.1, segments=None, save=False)}}
\label{fun:showhistogram}
\addcontentsline{toc}{subsection}{show\_histogram}

\begin{lstlisting}[language=docstring]
plots a histogram of a result parameter

Parameters
----------
x : str,
    name of a scalar from results
bins : int, optional
    number of bins, by default 50
cut : int, optional
    cut upper and lower percentile, by default 0.1
segments : list of int, optional
    list of segments or None for all data, by default None
save : bool/str, optional
    saves image with specified file extension, e.g. 'png', 'pdf'
    if True uses png, by default False
\end{lstlisting}

\begin{flushright}

\hyperref[toc]{ToC}

\end{flushright}

\input{functions/show_histogram}

\vspace{5mm}

\hrule

\subsection*{\texttt{show\_correlations(x, y, nbins=50, cut=(0.1, 0.1), segments=None, save=False)}}
\label{fun:showcorrelations}
\addcontentsline{toc}{subsection}{show\_correlations}

\begin{lstlisting}[language=docstring]
Plots a 2D histogram between 2 result parameters

Parameters
----------
x : str,
    name of a scalar from results
y : str,
    name of a scalar from results
bins : int, optional
    number of bins, by default 50
cut : tuple, optional
    cut upper and lower percentile of both parameters, by default (0.1,0.1)
segments : list, optional
    list of segments or None for all data, by default None
save : bool/str, optional
    saves image with specified file extension, e.g. 'png', 'pdf'
    if True uses png, by default False
\end{lstlisting}

\begin{flushright}

\hyperref[toc]{ToC}

\end{flushright}

\input{functions/show_correlations}

\vspace{5mm}

\hrule

\subsection*{\texttt{show\_voxel\_odf(x, y, z, num\_samples=1000)}}
\label{fun:showvoxelodf}
\addcontentsline{toc}{subsection}{show\_voxel\_odf}

\begin{lstlisting}[language=docstring]
Show a 3D plot of the ODF in the chosen voxel

Parameters
----------
x : int
    voxel x-coordinate
y : int
    voxel y-coordinate
z : int
    voxel z-coordinate
num_samples : int/float, optional
    number of samples for plot generation, by default 1000
\end{lstlisting}

\begin{flushright}

\hyperref[toc]{ToC}

\end{flushright}

\input{functions/show_voxel_odf}

\vspace{5mm}

\hrule

\subsection*{\texttt{show\_voxel\_polefigure(x, y, z, hkl=(1, 0, 0), mode='density', alpha=0.1, num\_samples=10000.0)}}
\label{fun:showvoxelpolefigure}
\addcontentsline{toc}{subsection}{show\_voxel\_polefigure}

\begin{lstlisting}[language=docstring]
Show a polefigure plot for the chosen voxel and hkl

Parameters
----------
x : int
    voxel x-coordinate
y : int
    voxel y-coordinate
z : int
    voxel z-coordinate
hkl : tuple, optional
    Miller indices, by default (1,0,0)
mode : str, optional
    plotting style 'scatter' or 'density', by default 'density'
alpha : float, optional
    opacity of points, only for scatter, by default 0.1
num_samples : int/float, optional
    number of samples for plot generation, by default 1e4
\end{lstlisting}

\begin{flushright}

\hyperref[toc]{ToC}

\end{flushright}

\input{functions/show_voxel_polefigure}

\vspace{5mm}

\hrule

\subsection*{\texttt{save\_results()}}
\label{fun:saveresults}
\addcontentsline{toc}{subsection}{save\_results}

\begin{lstlisting}[language=docstring]
Saves the results dictionary to a h5 file in the results/ directory

    
\end{lstlisting}

\begin{flushright}

\hyperref[toc]{ToC}

\end{flushright}

\input{functions/save_results}

\vspace{5mm}

\hrule

\subsection*{\texttt{list\_results()}}
\label{fun:listresults}
\addcontentsline{toc}{subsection}{list\_results}

\begin{lstlisting}[language=docstring]
Shows all results .h5 files in results directory
    
\end{lstlisting}

\begin{flushright}

\hyperref[toc]{ToC}

\end{flushright}

\input{functions/list_results}

\vspace{5mm}

\hrule

\subsection*{\texttt{load\_results(h5path='last', make\_bg\_nan=False)}}
\label{fun:loadresults}
\addcontentsline{toc}{subsection}{load\_results}

\begin{lstlisting}[language=docstring]
Loads the results from a h5 file do the results dictionary

Parameters
----------
h5path : str, optional
    filepath, just filename or full path
    if 'last', uses the youngest file is used in results/, 
    by default 'last'
make_bg_nan : bool, optional
    if true, replaces all excluded voxels by NaN
\end{lstlisting}

\begin{flushright}

\hyperref[toc]{ToC}

\end{flushright}

\input{functions/load_results}

\vspace{5mm}

\hrule

\subsection*{\texttt{list\_results\_loaded()}}
\label{fun:listresultsloaded}
\addcontentsline{toc}{subsection}{list\_results\_loaded}

\begin{lstlisting}[language=docstring]
Shows all results currently in memory
    
\end{lstlisting}

\begin{flushright}

\hyperref[toc]{ToC}

\end{flushright}

\input{functions/list_results_loaded}

\vspace{5mm}

\hrule

\subsection*{\texttt{save\_images(x, ext='raw')}}
\label{fun:saveimages}
\addcontentsline{toc}{subsection}{save\_images}

\begin{lstlisting}[language=docstring]
Export results as .raw or .tiff files for dragonfly

Parameters
----------
x : str,
    name of a scalar from results, e.g. 'scaling'
ext : str,
    desired file type by extension, can do 'raw' or 'tiff', default: 'raw'
\end{lstlisting}

\begin{flushright}

\hyperref[toc]{ToC}

\end{flushright}

\input{functions/save_images}

\vspace{5mm}

\hrule

\subsection*{\texttt{help(method=None, module=None, filter='')}}
\label{fun:help}
\addcontentsline{toc}{subsection}{help}

\begin{lstlisting}[language=docstring]
Prints information about functions in this library

Parameters
----------
method : str or None, optional
    get more information about a function or None for overview over all functions, by default None
module : str or None, optional
    choose python module or None for the base TexTOM library, by default None
filter : str, optional
    filter the displayed functions by a substring, by default ''
\end{lstlisting}

\begin{flushright}

\hyperref[toc]{ToC}

\end{flushright}

\input{functions/help}

\vspace{5mm}

\hrule

\end{document}\end{document}