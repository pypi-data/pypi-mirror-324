% arara: pdflatex

% start with vim --server latex %
\documentclass[]{article}

\usepackage{siunitx}
\usepackage{physics}
\usepackage{graphicx}
\usepackage{fullpage}

\author{C.D. Clark III}
\title{On...}
\begin{document}
\maketitle

% setup pint for automatic unit conversions
% {{{
% import pint
% ureg = pint.UnitRegistry()
% Q_ = ureg.Quantity
% }}}

% add a jinja2 fillter to format units with siunitx
% compudoc already defines a `fmt` filter that calls a function named
% `fmt_filter`, so we can just call that wit the Lx format
% modifier
% {{{
% def Lx_filter(input,fmt=""):
%   text = fmt_filter(input,fmt+"Lx")
%   return text
% jinja2_env.filters["Lx"] = Lx_filter
% }}}

Laser exposures are characterized by a power ($\Phi$), energy ($Q$), radiant exposure ($H$),
or irradiance ($E$). Each of these four radiometric quantities are related to each other
through the exposure area and duration.

% define some quantities to use
% {{{
% power = Q_(100,'mW')
% duration = Q_(0.25,'s')
% energy = (power * duration).to("mJ")
% }}}

For example, if a laser outputs a power of {{power | Lx}} for a
duration of {{duration | Lx}}, then the energy delivered during the
exposure will be {{energy | Lx}}, or {{energy.to("J")|Lx}}.

\end{document}
