\documentclass[12pt]{article}

\usepackage{amsmath}

\begin{document}

\title{Conversion between \texttt{pyFAI} and \texttt{ImageD11}
  detector geometry parameters} \author{J. Kieffer and C. Detlefs}
\date{\today} \maketitle

\section{Introduction}

The purpose of this note is to compare how \texttt{pyFAI} and
\texttt{ImageD11} treat the detector position. In particular, we
derive how ``PONI'' detector parameters refined with \texttt{pyFAI}
can be transformed into \texttt{ImageD11} parameters.

In both packages, the transformation from pixel space to 3D laboratory
coordinates is carried out in 4 steps:
\begin{itemize}
\item Transformation from ``pixel space'' to the ``detector
  coordinate system''. The detector coordinate system is a 3D
  coordinate system centered on the (0,0) pixel of the detector.
\item Correction for linear offsets, i.e.~the position of the (0,0)
  pixel relative to the beam axis.
\item Correction for the origin/diffractometer-to-detector
  distance. The sample and diffractometer center of rotation are
  assumed to be located at the origin.
\item A series of rotations for the detector coordinate system
  relative to the laboratory coordinates.
\end{itemize}

Unfortunately, the conventions chosen by \texttt{pyFAI} and
\texttt{ImageD11} differ. For example, \texttt{pyFAI} applies the
origin-to-detector distance correction before rotations, whereas
\texttt{ImageD11} applies it after rotations. Furthermore, they employ
different coordinate systems.

\section{Detector}

We consider a pixelated 2D imaging detector. In ``pixel space'', the
position of a given pixel is given by the horizontal and vertical
pixel numbers, $d_H$ and $d_V$. We assume that looking along the beam
axis into the detector, $d_H$ increases towards the right (towards the
center of the synchrotron) and $d_V$ towards the top. For clarity, we
assign the unit $\mathrm{px}$ to these coordinates.

The pixel numbers $d_H$ and $d_V$ are transformed into 3D ``detector''
coordinates by a function $D$:
\begin{align}
  \vec{p}
  & =
  D\left(d_H, d_V\right).
\end{align}
This function will account for the detector's pixel size and the
orientation and direction of pixel rows and columns relative to the
detector coordinate system. Furthermore it may apply a distortion
correction. This, however, is beyond the scope of this note.

Limiting ourselves to linear functions, $D$ takes the form of a matrix
with two columns and three rows. We will see below that the different
choices of laboratory coordinate systems yield different $D$-matrices
for \texttt{pyFAI} and \texttt{ImageD11}. We assume that the pixels
have a constant horizontal and vertical size, $\mathrm{pxsize}_H$ and
$\mathrm{pxsize}_V$. Both are given in units of length per
pixel. \texttt{pyFAI} specifically defines the unit of length as
meter, we will therefore use pixel sizes in units of
$\mathrm{m}/\mathrm{px}$ throughout this note.

The position and orientation of this detector relative to the
laboratory coordinates are described below.

\section{Geometry definition of \texttt{pyFAI}}

\subsection{Coordinates}

\texttt{pyFAI} uses a coordinate system where the first axis (1) is
vertically up ($y$), the second axis (2) is horizontal ($x$)
towards the ring center (starboard), and the third axis (3) along the
beam ($z$). Note that in this order (1, 2, 3) is a right-handed coordinate
system, which makes $xyz$ in the usual order a left-handed coordinate
system!

\subsection{Units}

All dimensions in \texttt{pyFAI} are in meter and all rotation are in
radians.

\subsection{Parameters}

\texttt{pyFAI} describes the position and orientation of the detector
by six variables, collectively called the PONI, for point of normal
incidence. In addition, a detector calibration is provided in the
PONI-file to convert pixel coordinates into real-space
coordinates. Here we limit our discussion to the simplest case, i.e.~a
pixel size as discussed above.

\begin{description}
\item[Rotations:] $\theta_1$, $\theta_2$ and $\theta_3$ describe the
  detector's orientation relative to the laboratory coordinate system.

\item[Offsets:] $\mathrm{poni}_1$ and $\mathrm{poni}_2$ describe the
  offsets of pixel (0,0) relative to the ``point of normal
  incidence''. In the absence of rotations the point of normal
  incidence is defined by the intersection of the direct beam beam
  axis with the detector.

\item[Distance:] $L$ describes the distance from the origin of the
  laboratory system to the point of normal incidence.
\end{description}

\subsection{Detector}

The transformation from pixel space to \texttt{pyFAI} detector
coordinates is given by

\begin{align}
  \begin{bmatrix} p_1 \\ p_2 \\ p_3 \end{bmatrix}
  & =
  \begin{bmatrix}
    0 & \mathrm{pxsize}_V \\
    \mathrm{pxsize}_H & 0 \\
    0 & 0
  \end{bmatrix}
  \cdot
  \begin{bmatrix} d_H \\ d_V \end{bmatrix}
  \\
  D_{\mathtt{pyFAI}}
  & =
  \begin{bmatrix}
    0 & \mathrm{pxsize}_V \\
    \mathrm{pxsize}_H & 0 \\
    0 & 0
  \end{bmatrix}.
  \label{eq-dmatrixpyFAI}
\end{align}

\subsection{Offsets}

The PONI parameters are: a distance $L$, the vertical ($y$) and
horizontal ($x$) coordinates of the point of normal incidence in
meters, $\mathrm{poni}_1$ and $\mathrm{poni}_2$. The inversion of the
$x$ and $y$ axes is due to the arrangement of the detector data, with
$x$-rows being the ``slow'' axis and $y$-columns the ``fast''
axis. Extra care has to be taken with the signs of the rotations when
converting form this coordinate system to another.

\texttt{pyFAI} applies both the offset correction and the
origin-to-detector distance after the transformation from pixel space
to the detector system, but before rotations,

Let $L$ be the distance from the origin/sample/diffractometer center
of rotation. In the absence of any detector rotations, $L$ is taken
along $p_3$ (beam axis, $z$), $p_1$ along the $y$-axis (vertical)
and $p_2$ along the $x$-axis (horizontal). Then the laboratory
coordinates before rotation are

\begin{align}
  \begin{bmatrix}
    p_1 \\ p_2 \\ p_3
  \end{bmatrix}
  & =
  D_{\mathtt{pyFAI}} \cdot \begin{bmatrix} d_H \\ d_V \end{bmatrix}
  +
  \begin{bmatrix} -\mathrm{poni}_1 \\ -\mathrm{poni}_2 \\ L \end{bmatrix}.
\end{align}

\subsection{Rotations}

The detector rotations are taken about the origin of the coordinate
system (sample position). We define the following right-handed
rotation matrices:

\begin{align}
  \mathrm{R}_1(\theta_1)
  & =
  \begin{bmatrix}
    1 & 0 & 0 \\
    0 & \cos(\theta_1) & -\sin(\theta_1) \\
    0 & \sin(\theta_1) & \cos(\theta_1)
  \end{bmatrix}
  \label{eq-rot1}
  \\
  \mathrm{R}_2(\theta_2)
  & =
  \begin{bmatrix}
    \cos(\theta_2) & 0 & \sin(\theta_2) \\
    0 & 1 & 0 \\
    -\sin(\theta_2) & 0 & \cos(\theta_2)
  \end{bmatrix}
  \label{eq-rot2}
  \\
  \mathrm{R}_3(\theta_3)
  & =
  \begin{bmatrix}
    \cos(\theta_3) & -\sin(\theta_3) & 0\\
    \sin(\theta_3) & \cos(\theta_3) & 0\\
    0 & 0 & 1
  \end{bmatrix}.
  \label{eq-rot3}
\end{align}

The rotations 1 and 2 in \texttt{pyFAI} are left handed, i.e.~the sign
of $\theta_1$ and $\theta_2$ is inverted.

The combined \texttt{pyFAI} rotation matrix is then
\begin{align}
  R_{\mathtt{pyFAI}}(\theta_1, \theta_2, \theta_3)
  & =
  R_3(\theta_3) \cdot R_2(-\theta_2) \cdot R_1(-\theta_1)
\end{align}

which yields the final  laboratory coordinates after rotation

\begin{align}
  \begin{bmatrix}
    t_1 \\ t_2 \\ t_3
  \end{bmatrix}
  & =
  R_{\mathtt{pyFAI}}(\theta_1, \theta_2, \theta_3)
  \cdot
  \begin{bmatrix} p_1 \\ p_2 \\ p_3 \end{bmatrix}
  \label{eq-tpyFAI}
  \\
  & =
  R_{\mathtt{pyFAI}}(\theta_1, \theta_2, \theta_3)
  \cdot
  \left(
  D_{\mathtt{pyFAI}} \cdot \begin{bmatrix} d_H \\ d_V \end{bmatrix}
  + \begin{bmatrix} -\mathrm{poni}_1 \\ -\mathrm{poni}_2 \\ L \end{bmatrix}
  \right)
  \\
  & =
  R_{\mathtt{pyFAI}}(\theta_1, \theta_2, \theta_3)
  \cdot
  \left(
  \begin{bmatrix}
    0 & \mathrm{pxsize}_V \\
    \mathrm{pxsize}_H & 0 \\
    0 & 0
  \end{bmatrix}
  \cdot \begin{bmatrix} d_H \\ d_V \end{bmatrix}
  + \begin{bmatrix} -\mathrm{poni}_1 \\ -\mathrm{poni}_2 \\ L \end{bmatrix}
  \right).
\end{align}

\subsection{Inversion: Finding where a scattered beam hits the detector}

For a 3DXRD-type simulation, we have to determine the pixel where a
scattered ray intercepts the detector. Let $A$ be the scattering
center of a ray within a sample volume (grain, sub-grain or
voxel). The Bragg condition and grain orientation pre-define the
direction of the scattered beam, $\vec{k}$. The coordinates
$A_{1,2,3}$ and $k_{1,2,3}$ are specified in the laboratory system.

The inversion eq.~\ref{eq-tpyFAI} is straight-forward:

\begin{align}
  R_1(\theta_1)\cdot R_2(\theta_2) \cdot R_3(-\theta_3) \cdot
  \begin{bmatrix} t_1 \\ t_2 \\ t_3
  \end{bmatrix}
  & =
  \begin{bmatrix} p_1 \\ p_2 \\ L \end{bmatrix}
  \label{eq-find-alpha}
  \\
  \begin{bmatrix}
    t_1 \\ t_2 \\ t_3
  \end{bmatrix}
  & =
  \begin{bmatrix}
    A_1  \\ A_2 \\ A_3
  \end{bmatrix}
  + \alpha
  \begin{bmatrix}
    k_1 \\ k_2 \\ k_3
  \end{bmatrix}.
\end{align}

The third line ($\ldots = L$) of eq.~\ref{eq-find-alpha} is then used
to determine the free parameter $\alpha$, which in turn is used in the
first and second lines to find $p_{1,2}$ and thus $d_{1,2}$.

As the most trivial example we consider the case of no rotations,
$\theta_1 = \theta_2 = \theta_3 = 0$. Then

\begin{align}
  A_3 + \alpha k_3 & = L \\
  \alpha & = \frac{L-A_3}{k_3} \\
  p_1 & = A_1 + (L-A_3) \frac{k_1}{k_3} \\
  p_2 & = A_2 + (L-A_3) \frac{k_2}{k_3}.
\end{align}

We see also that when all rotations are zero, $(\mathrm{poni}_1,
\mathrm{poni_2})$ are the real space coordinates of the direct beam
($A_{1,2,3}=k_{1,2}=0$) .

\section{Geometry definition of \texttt{ImageD11}}

For maximum convenience, \texttt{ImageD11} defines almost everything
differently than \texttt{pyFAI}.

\subsection{Coordinates}

\texttt{ImageD11} uses the ID06 coordinate system
with $x$ along the beam, $y$ to port (away from the ring center), and
$z$ up.

\subsection{Units}

As the problem is somewhat scale-invariant, \texttt{ImageD11}
allows a free choice of the unit of length, which we will call $X$
here. The same unit has to be used for all translations, and for the
pixel size of the detector. The default used in the code appears to be
$X = 1\,\mathrm{\mu m}$, but it might as well be Planck lengths,
millimeters, inches, meters, tlalcuahuitl, furlongs, nautical miles,
light years, kparsec, or whatever else floats your boat. The only
requirement is that you can actually measure and express the detector
pixel size and COR-to-detector distance in your units of choice. Since
we want to compare to \texttt{pyFAI}, we choose $X=1\,\mathrm{m}$.

Rotations are given in radians.

\subsection{Parameters}

\texttt{ImageD11} defines the detector geometry via the
following parameters:

\begin{description}
\item[Beam center:] $y_{\mathrm{center}}$ and $z_{\mathrm{center}}$
  define the position of the direct beam on the detector. Contrary to
  \texttt{pyFAI}, the beam center is given in pixel space, in units of
  $\mathrm{px}$.

\item[Pixel size:] The horizontal and vertical pixel size are defined
  by $y_{\mathrm{size}}$ and $z_{\mathrm{size}}$ in
  ${X}/{\mathrm{px}}$. With the right choice of the unit of length
  $X$, these corresponds directly to the pixel sizes
  $\mathrm{pxsize}_H$ and $\mathrm{pxsize}_V$ defined above.

\item[Detector flip matrix:] $O = \begin{bmatrix} o_{11} & o_{12}
  \\ o_{21} & o_{22} \end{bmatrix}$. This matrix takes care of
  correcting typical problems with the way pixel coordinates are
  arranged on the detector. If, e.g., the detector is rotated by
  $90^{\circ}$, then $O=\begin{bmatrix} 0 & 1 \\ -1 &
  0\end{bmatrix}$. If left and right (or up and down) are inverted on
  the detector, then $o_{22} = -1$ ($o_{11}=-1$).

\item[Rotations:] Detector tilts $t_x$, $t_y$, and $t_z$, in
  $\mathrm{rad}$. The center of rotation is the point where the direct
  beam intersects the detector.

\item[Distance:] $\Delta$, in units $X$, is the distance between the origin
  to the point where the direct beam intersects the detector.  Note
  that this is again different from the definition of \texttt{pyFAI}.
\end{description}

It appears that these conventions where defined under the assumption
that the detector is more or less centered in the direct beam, and
that the detector tilts are small.

\subsection{Transformation}

The implementation in the code \texttt{transform.py} is using the
following equations:

\begin{align}
  R_{\mathtt{ImageD11}}(\theta_x, \theta_y, \theta_z)
  & =
  R_1(\theta_x) \cdot R_2(\theta_y) \cdot R_3(\theta_z)
  \\
  \begin{bmatrix}
    p_z \\ p_y
  \end{bmatrix}
  & =
  \begin{bmatrix}
    o_{11} & o_{12}
    \\ o_{21} & o_{22}
  \end{bmatrix}
  \cdot
  \begin{bmatrix}
    (d_z - z_{\mathrm{center}}) z_{\mathrm{size}} \\
    (d_y - y_{\mathrm{center}}) y_{\mathrm{size}}
  \end{bmatrix}
  \label{eq-p}
  \\
  \begin{bmatrix}
    t_x \\ t_y \\ t_z
  \end{bmatrix}
  & =
  R_{\mathtt{ImageD11}}(\theta_x, \theta_y, \theta_z)
  \cdot
  \begin{bmatrix}
    0 \\ p_y \\ p_z
  \end{bmatrix}
  +
  \begin{bmatrix}
    \Delta \\ 0 \\ 0
  \end{bmatrix}
  \label{eq-tImageD11}
\end{align}
Note that the order of $y$ and $z$ is not the same in eqs.~\ref{eq-p}
and \ref{eq-tImageD11}.

By combining the detector flip matrix $O$ and the pixel size into a
detector $D$ matrix, this can be written as

\begin{align}
  D_{\mathtt{ImageD11}}
  & =
  \begin{bmatrix}
    0 & 0 \\
    y_{\mathrm{size}} o_{22} & z_{\mathrm{size}} o_{21} \\
    y_{\mathrm{size}} o_{12} & z_{\mathrm{size}} o_{11}
  \end{bmatrix}
  \label{eq-DImageD11}
  \\
  \begin{bmatrix} p_x \\ p_y \\ p_z \end{bmatrix}
  & =
  D_{\mathtt{ImageD11}} \cdot
  \begin{bmatrix}
    d_H - y_{\mathrm{center}} \\
    d_V - z_{\mathrm{center}}
  \end{bmatrix}
\end{align}

\section{Conversion}

Assume that the same detector geometry is described by the two
notations. How can the parameters be converted from one to the other?

\subsection{Detector $D$-matrix}

The pixel size is the same in both notations, $y_{\mathrm{size}} =
\mathrm{pxsize}_H$ and $z_{\mathrm{size}} = \mathrm{pxsize}_V$.

As \texttt{pyFAI} does not allow for detector flipping, $o_{11}=1$,
$o_{22}=-1$ (because the sign of the horizontal axis is inverted
between \texttt{ImageD11} and \texttt{pyFAI}) and $o_{12}=o_{21}=0$.
For the detector setup described above, with $d_V$ increasing to the
top and $d_H$ increasing towards the center of the synchrotron
(i.e.~opposite to the positive $y$-direction), eq.~\ref{eq-DImageD11}
becomes

\begin{align}
  D_{\mathtt{ImageD11}}
  & =
  \begin{bmatrix}
    0 & 0 \\ -\mathrm{pxsize}_H & 0 \\ 0 & \mathrm{pxsize}_V
  \end{bmatrix}.
  \label{eq-dmatrixImageD11}
\end{align}

\subsection{Coordinates}

Both notations use the same sign for the vertical and beam axes. The
sign of the horizontal transverse axis, however, is inverted.

The transformation between the different coordinate systems is then
achieved by
\begin{align}
  G & =
  \begin{bmatrix}
    0 & 0 & 1 \\ 0 & -1 & 0 \\ 1 & 0 & 0
  \end{bmatrix}
  \\
  t_{\mathtt{ImageD11}}
  & =
  G \cdot
  t_{\mathtt{pyFAI}},
  \label{eq-coordconv}
\end{align}
where $t_{\mathtt{ImageD11}}$ is given by eq.~\ref{eq-tImageD11}, and
$t_{\mathtt{pyFAI}}$ is given by eq.~\ref{eq-tpyFAI}. The matrix $G$
performs the change of axes ($x \leftrightarrow z$, $y \leftrightarrow
-y$) and has the convenient property $G^2 = 1$.

Substituting these equations into eq.~\ref{eq-coordconv}, one can them
attempt to convert \texttt{pyFAI} parameters into \texttt{ImageD11}
parameters and vice versa.

\begin{align}
  R_{\mathtt{ImageD11}}
  \cdot
  D_{\mathtt{ImageD11}}
  &
  \cdot
  \begin{bmatrix}
    d_H - y_{\mathrm{center}} \\
    d_V - z_{\mathrm{center}}
  \end{bmatrix}
  +
  \begin{bmatrix} \Delta \\ 0 \\ 0 \end{bmatrix}
  \nonumber \\
  = &
  G \cdot
  R_{\mathtt{pyFAI}}
  \cdot
  \left(
  D_{\mathtt{pyFAI}}
  \cdot
  \begin{bmatrix} d_H \\ d_V \end{bmatrix}
  +
  \begin{bmatrix} -\mathrm{poni}_1 \\ -\mathrm{poni}_2 \\ L \end{bmatrix}
  \right)
  \label{eq-transformation}
\end{align}

\subsection{Rotations}

Take an arbitrary vector $d$ with $d_{\mathtt{ImageD11}}
= \begin{bmatrix} a \\ b \\ c \end{bmatrix}$. We first transform this
into the \texttt{pyFAI} coordinate system by multiplication with $G$,
and then apply an arbitrary rotation matrix, once in before (in
\texttt{pyFAI} coordinates, $R_{\mathtt{pyFAI}}$) and once after the
transformation (in \texttt{ImageD11} coordinates,
$R_{\mathtt{ImageD11}}$).

\begin{align}
    d_{\mathtt{pyFAI}}
    & =
    G \cdot d_{\mathtt{ImageD11}}
    = \begin{bmatrix} c \\ -b \\ a \end{bmatrix}
    \\
    R_{\mathtt{pyFAI}} \cdot d_{\mathtt{pyFAI}}
    & =
    R_{\mathtt{pyFAI}} \cdot G \cdot d_{\mathtt{ImageD11}}
    \\
    & = G \cdot R_{\mathtt{ImageD11}} \cdot d_{\mathtt{ImageD11}}.
\end{align}

Comparing the last two lines, we find that with

\begin{align}
  R_{\mathtt{pyFAI}} \cdot G
  & =
  G \cdot R_{\mathtt{ImageD11}}
\end{align}
the transformation is applicable for each and any vector $d$.  Because
$G^{-1} = G$ this transformation can also be applied to a series of
rotations: $G \cdot R \cdot R' = (G \cdot R \cdot G) \cdot (G \cdot R'
\cdot G) \cdot G$.

Applying this to the rotations matrices defined in
eqs.~\ref{eq-rot1}--\ref{eq-rot3} shows, unsurprisingly, that this
coordinate transformation is an exchange of rotation axes $x$ and
$y$, and a change of sign for $y$.

\begin{align}
  G \cdot R_1(\theta) \cdot G & = R_3(\theta) \\
  G \cdot R_2(\theta) \cdot G & = R_2(-\theta) \\
  G \cdot R_3(\theta) \cdot G & = R_1(\theta)
\end{align}

Applying this transformation to the \texttt{pyFAI} rotation matrix can
comparing to the \texttt{ImageD11} rotation matrix, we see
\begin{align}
  G \cdot R_{\mathtt{pyFAI}}(\theta_1, \theta_2, \theta_3)
  \cdot G
  & =
  G R_3(\theta_3) \cdot R_2(-\theta_2) \cdot R_1(-\theta_1)
  \cdot G
  \\
  & =
  R_1(\theta_3) \cdot R_2(\theta_2) \cdot R_3(-\theta_1)
  \\
  & =
  R_{\mathtt{ImageD11}}(\theta_x, \theta_y, \theta_z)
  \\
  & =
  R_1(\theta_x) \cdot R_2(\theta_y) \cdot R_3(-\theta_z)
\end{align}

We find that, by divine intervention\footnote{May his noodly
  appendages forever touch you!} and despite all the efforts to choose
incompatible conventions, \emph{the effective order of rotations is
  actually the same between \texttt{ImageD11} and
  \texttt{pyFAI}}. Consequently, there is a direct correspondence with
only a change of sign between $\theta_z$ and $\theta_1$:

\begin{align}
  \theta_x & = \theta_3
  \label{eq-thetax}
  \\
  \theta_y & = \theta_2
  \label{eq-thetay}
  \\
  \theta_z & = -\theta_1
  \label{eq-thetaz}
\end{align}

\subsection{Translations and offsets}

Inserting eqs.~\ref{eq-thetax}--\ref{eq-thetaz}
into \ref{eq-transformation}, we find

\begin{align}
  \begin{bmatrix} \Delta \\ 0 \\ 0 \end{bmatrix}
  = &
  G \cdot
  R_{\mathtt{pyFAI}}
  \cdot
  \left(
  D_{\mathtt{pyFAI}}
  \cdot
  \begin{bmatrix} d_H \\ d_V \end{bmatrix}
  +
  \begin{bmatrix} -\mathrm{poni}_1 \\ -\mathrm{poni}_2 \\ L \end{bmatrix}
  \right)
  \nonumber \\ &
  -
  R_{\mathtt{ImageD11}}
  \cdot
  D_{\mathtt{ImageD11}}
  \cdot
  \begin{bmatrix}
    d_H - y_{\mathrm{center}} \\
    d_V - z_{\mathrm{center}}
  \end{bmatrix}
  \\
  = &
  R_{\mathtt{ImageD11}}
  \cdot
  G \cdot
  \left(
  \begin{bmatrix}
    \mathrm{pxsize}_V d_V \\ \mathrm{pxsize}_H d_H \\ 0
  \end{bmatrix}
  +
  \begin{bmatrix} -\mathrm{poni}_1 \\ -\mathrm{poni}_2 \\ L \end{bmatrix}
  \right)
  \nonumber \\ &
  -
  R_{\mathtt{ImageD11}}
  \cdot
  \begin{bmatrix}
    0 \\
    -\mathrm{pxsize}_H (d_H - y_{\mathrm{center}}) \\
    \mathrm{pxsize}_V (d_V - z_{\mathrm{center}})
  \end{bmatrix}
  \\
  = &
  R_{\mathtt{ImageD11}}
  \cdot
  \left(
  \begin{bmatrix}
    L
    \\
    \mathrm{poni}_2 - \mathrm{pxsize}_H d_H
    \\
    -\mathrm{poni}_1 + \mathrm{pxsize}_V d_V
  \end{bmatrix}
  -
  \begin{bmatrix}
    0 \\
    -\mathrm{pxsize}_H (d_H - y_{\mathrm{center}}) \\
    \mathrm{pxsize}_V (d_V - z_{\mathrm{center}})
  \end{bmatrix}
  \right)
  \\
  = &
  R_{\mathtt{ImageD11}}
  \cdot
  \begin{bmatrix}
    L
    \\
    \mathrm{poni}_2 - \mathrm{pxsize}_H y_{\mathrm{center}}
    \\
    -\mathrm{poni}_1 + \mathrm{pxsize}_V z_{\mathrm{center}}
  \end{bmatrix}.
\end{align}

%For the simplest case of all rotations being zero, $\Delta = L$,
%$y_{\mathrm{center}} = \mathrm{poni}_2 / \mathrm{pxsize}_H$, and
%$z_{\mathrm{center}} = -\mathrm{poni}_1 / \mathrm{pxsize}_V$.

With a little help from our friend Mathematica, we find for the
conversion from \texttt{pyFAI} to \texttt{ImageD11}

\begin{align}
  \Delta
  & =
  \frac{L}{\cos(\theta_1) \cos(\theta_2)}
  \\
  y_{\mathrm{center}}
  & =
  \frac{1}{\mathrm{pxsize}_H}
  \left(
  \mathrm{poni}_2 - L \tan(\theta_1)
  \right)
  \\
  z_{\mathrm{center}}
  & =
  \frac{1}{\mathrm{pxsize}_V}
  \left(
  \mathrm{poni}_1 + L \frac{\tan(\theta_2)}{\cos(\theta_1)}
  \right),
\end{align}

and for the conversion from \texttt{ImageD11} to \texttt{pyFAI}

\begin{align}
  L
  & =
  \Delta \cos(\theta_y) \cos(\theta_z)
  \\
  \mathrm{poni}_1
  & =
  -\Delta \sin(\theta_y) + \mathrm{pxsize}_V z_{\mathrm{center}}
  \\
  \mathrm{poni}_2
  & =
  -\Delta \cos(\theta_y) \sin(\theta_z) + \mathrm{pxsize}_H y_{\mathrm{center}}.
\end{align}

\end{document}
